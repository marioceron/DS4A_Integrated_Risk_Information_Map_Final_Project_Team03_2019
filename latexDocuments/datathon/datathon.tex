\documentclass[11pt]{article}

\usepackage[margin = 0.5 in]{geometry}
\usepackage[pdftex]{graphicx}
\graphicspath{{images/}}
\usepackage{subfigure}
\usepackage{booktabs}
\usepackage{multirow}
\usepackage[table]{xcolor}
\usepackage{array}
%\usepackage{amsmath}
\usepackage{booktabs}
\usepackage{multirow}
\usepackage{amsmath}
\usepackage{url}
\usepackage[colorlinks=true,urlcolor=blue]{hyperref}
\renewcommand\UrlFont{\color{blue}}


\title{Datathon Group 3}
\author{ Johnathan Salamanca, Mario Cer\'{o}n, \\
Carol Martinez, Javier Cocunubo, Jairo Nino, Alvaro Munoz
 }
\date{\today}

\begin{document}
\maketitle

\begin{abstract}

In this document the project scope and plan for the Datathon are presented. The document provides information of the data cleaning process and some plots with preliminar results of the data wrangling process.


\end{abstract}

%\section{Introduction}
%You are asked to explore the
%provided data and based on this, pose a question which you believe would be interesting to
%answer. You and your team will then write a report detailing why you believe this question is
%important and how you went about answering it.


\section{Project Scoping \& Plan}
\label{sec:app}

\subsection{Scope}


\begin{itemize}

\item \textbf{Project Objective:}
\begin{itemize}
\item \textbf{Main question}: \textit{How do yellow cabs mean trip distance have changed over time (rush/non-rush hours) as a result of Uber's trips growth?} \\ 
\textit{What are the patterns related to unattended areas of public service?} \\ 

%sFrom this one we can analyze the mean income of the zones where yellow cabs drop-off zones changed.
\end{itemize}

\item \textbf{Main stakeholders}: the NYC citizen and government, and transportation industry (at all levels).

\item \textbf{Boundaries of the project}:

\begin{itemize}
%\item We do not offer forecasts or modeling/ infer data.
\item We will show metrics of the impact of Uber incursion in NYC over the other transportation means.
\item The analysis will be made only on the information of the NYC Boroughs.
%\item We do not offer recommendations, only do support to decision making process for the stakeholders.
\end{itemize}

\item \textbf{Risks:} 
\begin{itemize}
\item Data quality issues in the datasets.
\item The data might be not sufficient to answer the proposed question. 
\end{itemize}

\end{itemize}


\subsection{Plan}
\begin{itemize}
\item \textbf{Summary}: \textit{How do yellow cabs mean trip distance have changed over time (rush/non-rush hours) as a result of Uber's trips growth?}  From this one we can analyze the mean income of the zones where yellow cabs drop-off zones changed.


\item \textbf{Expected Deliverable}: A report with the topic question, Data wrangling and Cleaning process, Exploratory Data Analysis EDA, Statistical Analysis and Modeling, Results Interpretation and Conclusions.

\item \textbf{How to get there}: 
\begin{itemize}
\item Clean, wrangled and analyze the dataset.
\item Conduct exploratory data analysis.
\item Conduct Analysis \& modeling.
\item Conclusions and final report (source code and power point presentation).
\end{itemize}

\end{itemize}

%\section{Topic Question}
%\abel{sec:topicQ}
%What is the question that your team set out to answer? How did you
%come about to choosing this question, and why is it an important question?
\section{Introduction}
\label{intro}
This document describes the process followed to define, analyze and answer  the questions of the Datathon project. The data provided correspond to datasets from 2014 and 2015 of four different transportation means: Yellow cabs, Green cabs, UBER, and MTA. Additionally, one dataset provides weather information, and another one provides demographic information of the boroughs. 


\subsection{Background Information}
\label{intro_background}

New York city is the most populous city in the United States \cite{wikiNYC} with around 8.6 million inhabitants. It is expected to reach 9 million by 2040 \cite{growthNYC}, with Bronx being the borough with the highest increase in population with 14\% (between 2010 and 2040). Average Travel speeds in New York City is 10 mph, 4 mph or less in Brooklyn and Queens, and over 3 mph NYC Midtown. 


In terms of transportation, according to \cite{trafficNYC2}, New York City is the second most congested city in the USA and number 42 in the world. It has one of the largest and oldest (1904) subway systems in the world. Due to its congestion, New York City inhabitants prefer to use public transportation ($67.2\%$ of workers commuting to work by this means in 2006 \cite{wikiNYC})

However, MTA the worst commutes in the world \cite{commuteWorld} conmute time of 35.6 minutes on average commuting has been linked to obesity, stress, anxiety, depression, higher blood pressure, higher rates of divorce, neck and back pain and shorter lifespans.





%In Midtown taxi trips are short distances... comparable ttrips by citi bikes are generally faster and less expensive...  between 1-1.5 miles are more than 5 minutes faster and 10 USD cheaper than taxi.. 
%
%Recent Travel trends:
%- Number of trips of yellow cabs drop by XXX
%- number o for hire vehicle registrations has grown (yellow cabs green cabs, black cabs, private cabs companies)..  72000 in 2015
% - MTA 
% -Higher speeds LESS dense residential areas of eastern queens and the northern sections of the Bronx Staten Island
% - Average taxi speed.
% 
% 
% between the 5 most congested cities in the USA one of the 
%It has one of the largest subway systems in the world. It is also famous because people does not like to own a car they prefer to use public transportation.. Highest rate of public transportation use of any american city $67.2\%$ of workers commuting to work by this means in 2006.. Longest mean travel time for commuters. Half of all households do not own a car. In manhattan around 75\% 

%2014 8.521 mllion
%2015  8.522 million
%2019  8.398.748 million

%New York City's population is expected to reach 9 million by 2040 \cite{growthNYC}, based on recent projections created by the city. Among the five boroughs, the Bronx's growth is projected to be the highest at 14\% between 2010 and 2040. On the flip side, Manhattan is expected to grow by 6.7\% by 2040. Median age are 36.2 years between male and female.. rate of home ownership 32.6\% Percentage in POVERTY POWERTY RATE.. EMPLOYMENT STATUS

%If these projections are accurate, Brooklyn will extend its lead over Queens as the largest borough in New York City, growing to nearly 3 million by 2040.
%
%
%NYC mass transit..  
%
%NYC This city houses the highest number of millionaires and billionaires in the world.
%
%High rate of publich transit use
%a city is the most mwith XXX inhabitants [ref]. It is commontly known for its transportation innovations 
%NYC is the third most traffic congested city in the world \cite{trafficNYC}
%
%
%3.7 million people were employed in New York City
%Manhattan 56\% of all jobs
%Of those working in Manhattan, 30% commute from within Manhattan, while 17% come from Queens, 16% from Brooklyn, 8% from the Bronx, and 2.5% from Staten Island. Another 4.5% commute to Manhattan from Nassau County and 2% from Suffolk County on Long Island, while 4% commute from Westchester County. 5% commute from Bergen and Hudson counties in New Jersey.
%NYC 8.5 million inhabitants.

%\textbf{future of NYC}
%Traffic has been always a problem in NYC 

%In terms of Yellow cabs, the following list summarizes key points of this transportation mode. 
The following lists summarize key points of the different transportation means used in the Datathon. \\

\textbf{Yellow Cabs}

\begin{itemize}
\item Mostly located in Manhattan
\item It is very difficult to get a taxi out of Manhattan, especially in rush hours.
\item There are not many yellow cabs 130000, which are not enough.
\item They can operate midtown and lower Manhattan and airports
\item Rarely pick up outside manhattan
\end{itemize}



\textbf{Green Cabs}
\begin{itemize}
\item Where created to standardize street hails outside of NYC. They operate in Brooklyn, Queens, The Bronx, and upper Manhattan. 
%\item Created to provide more access to metered taxis. Less expensive than livery cars.
%\item They operate Manhattan Bello 110th St on the west side and below 96th street on the east side or at either la guardia or JFK airports.

\item Airports: they can drop-off but not pick-up unless sent by a dispatcher
%\item They can \textbf{be on call by dispatcher}
\item They are not allowed to stop in the South of the upper west and upper east sides.
%\item The permits of Green cabs are cheaper than yellow cabs and easier to acquire. Cab license affordable for drivers.
\item Rides cheaper than Yellow cab rides
\item For drivers green cabs was a way to get money without the pressure that yellow drivers have.
\item Green drivers  \textbf{are also drivers or UBER}
\item 1/3 pick ups from Brooklyn,  1/3 Northern Manhattan, 1/3 Queens, a few in Bronx and Staten Island \\
\end{itemize}

%\subsubsection{Uber}


\textbf{UBER}

\begin{itemize}
\item Started in 2011 in Manhattan but expanded at the same time green.
\item Uber has made yellow cabs steady but has impacted green that were just started 
\item Uber has made yellow cabs steady but has impacted green that were just started 
\item Connects drivers with more rides
\item May 2015 busiest month on record.
\end{itemize}



According to the mobility report of 2016 \cite{mobilityReportNYC2}, New York City is growing in jobs, residents and visitors. Its transportations modes have extend to mass transit, walking, and cycling. New York advances focus on applying technology to real-time traffic management, Select Bus Service routes, reducing travel times, Expanding the city's bicycle lane network, improving pedestrian access to transit. It is still required to invest in ways to keep the city moving. 


Despite the incursion of new transportation means e.g. Green Cabs, UBER, among others, it is still difficult to catch a taxi from outer Boroughs to Manhattan. Lower access to legal taxi rides for people in outer Boroughs.
%\item Green cabs try but never fulfilled their promise


%\end{itemize}


\subsection{Topic Question}

\textbf{General question}: \\

%\begin{center}
%\tcbox[top=15pt,left=15pt,right=15pt,bottom=15pt, on line, fontupper=\Large]{ \textit{Is public transportation coverage in New York City \newline 
%\textbf{equally attending} all areas?}}
%\end{center}

\begin{tcolorbox}[colback=white,top=6pt,bottom=6pt, breakable,arc=3pt,
  outer arc=3pt]
\begin{center}
 \textit{Is public transportation coverage in New York City \\
\textbf{equally attending} all areas?\\}
\end{center}
\end{tcolorbox}

\vspace{0.5cm}

\textbf{Exploratory questions}: 
\begin{itemize}
\item What are the patterns related to unattended areas of public service?
\item Is there are relationship between demographic information and peoples' choices of public transportation?
\end{itemize}




%\begin{center}
%\begin{myboxi}[General question]
%\center
%\textit{Is public transportation coverage in New York City equally attending all areas?} \\ 
%\end{myboxi}
%\end{center}

\section{Data Wrangling and Data Cleaning}
\label{subsec:dataCl}


The data cleaning process was done in two steps:

\begin{itemize}
\item For yellow and green cap trips, the rows that have distances equal to $0$ we deleted. This, because we are aiming to take into account only the trips that traveled some distance.
\item For yellow and green caps trips, the IQR methodology was used to clean the outliers from the data. A variable called ``amount\_per\_distance'' was created. It was calculated as the ratio between ``total\_amount" and ``trip\_distance". With this new variable, the values that did not show a common relationship between distance and values were deleted.
\end{itemize}

Feature engineering:
We created a new variable that measures the ratio between the total amount of the trip and the distance it traveled. This feature was created for Yellow trips and Green trips and was used for the outlier cleansing.

\subsection{Data Analysis}

Different plots were created with the aim to understand the behaviour of the different transportation systems. The following plots summarize the important findings encountered, so far.

Figure \ref{fig:boxTrips} shows the graph plots of the monthly trips for the different transportation systems: Figure \ref{fig:a} for Uber's trips, Figure \ref{fig:b} for Yellow cap trips, Figure \ref{fig:c} for Green cap trips, and Figure \ref{fig:d} for MTA trips. The boxplots differentiate the trips between rush hours (orange boxes) and non-rush hours (blue boxes). From the figure it can be seen that the MTA is highly used in rush hours. Additionally, it is possible to see that there has been a significan increase of the number of trips taken by Uber from 2015 both in rush and non-rush hours; and a decrease on the number of trips taken by Yellow caps, especially in rush hours. 

\begin{figure}%
\centering
\subfigure[Uber trips.]{%
\label{fig:a}%
\includegraphics[height=2in, width=3.5in]{UBER_trips_only_weekdays_rush_and_non_rush.png}}%
\qquad
\subfigure[Yellow cap trips]{%
\label{fig:b}%
\includegraphics[height=2in, width=3.5in]{yellowTripsRushHours.png}}%
\qquad
\subfigure[Green cap trips]{%
\label{fig:c}%
\includegraphics[height=2.4in, width=3.5in]{green_trips_month_rush.png}}%
\qquad
\subfigure[MTA trips]{%
\label{fig:d}%
\includegraphics[height=2in, width=3.5in]{mta_new_entries_orange_rush_hour.png}}%
\caption{Monthly behaviour of the number of trips. Orange boxes represent the number of trips in rush hours and blue ones correspond to non-rush hours. }
\label{fig:boxTrips}%
\end{figure}



Figure \ref{fig:boxDistances} compares the monthly average travel distance covered by Uber and Yellow caps. From these plots, it is possible to see that from 2015, Uber is widely used for long distance trips, in contrast to Yellow caps. On the other hand, the average travel distance of Yellow caps experienced an increase in April 2015. 


\begin{figure}%
\centering
\subfigure[Uber]{%
\label{fig:first}%
\includegraphics[height=2in, width=5in]{UBER_Trips_per_month_only_weekends.png}}%
\qquad
\subfigure[Yellow caps]{%
\label{fig:second}%
\includegraphics[height=2.2in, width=5.5in]{Yellow_Cabs_Average_trip_distance_Weekend_Only_by_Month_copy.png}}%
\caption{Comparison of the monthly average travel distance covered by Uber and Yellow caps. From the plot  }
\label{fig:boxDistances}%
\end{figure}

\section{Exploratory Data Analysis}
\label{sec:datAnalisys}

\subsection{Hypothesis}

During the data analysis we stated different hypothesis:

\begin{itemize}

\item \textbf{H1}: Due to UBER incursion in the city, the \textbf{travel distances} of yellow cabs trips increased.

\item \textbf{H2}: Due to UBER incursion in the city, the \textbf{number of trips} of yellow cabs was reduced.

\item \textbf{H3}: The areas of the city where there is \textbf{no public transportation} coverage, especially for hire transportation, correspond to areas with \textbf{low income families}. 

\item \textbf{H4}: Due to UBER incursion in the city, the price of yellow cabs trips have decreased. 

\end{itemize}


\subsection{Data Exploration}

Different plots were created with the aim of understanding the behaviour of the different transportation systems. The following figures show some of the comparisons that were done, and shows the relationship of the plots with the previously mentioned hypothesis.

\subsubsection{General Plots}

Figure \ref{fig:boxTrips} shows the boxplots of the monthly number of trips for the different transportation systems. Figure \ref{fig:a} for Uber's trips, Figure \ref{fig:b} for Yellow cab trips, Figure \ref{fig:c} for Green cab trips, and Figure \ref{fig:d} for MTA trips. The boxplots differentiate the trips between rush hours (orange boxes) and non-rush hours (blue boxes). From the figures, it can be seen that the MTA is busier in rush hours than in non-rush hours. Additionally, it is possible to see that there has been a significant increase of the number of trips taken by Uber from 2015 both in rush and non-rush hours; and a decrease on the number of trips taken by Yellow cabs, especially in rush hours. The latter corresponds to the ideas stated in hypothesis \textbf{H3}.

Figure \ref{fig:boxTripsWeekends} shows the same plot, but separating weekdays (blue boxes) from weekends. In the figure we can also see the previously mentioned behaviour: UBER trips increase and yellow trips tend to decrease.

Additionally, Figure \ref{fig:boxTripsHour} was created to analyze if users prefer to use a specific service at a specific hour. We can see that pick ups are high when people are moving to work, i.e. 7 am (yellow, UBER, and green cabs have a high number of pick ups during that hour). Nevertheless, the highest number of pick ups occur between 6 and 7 pm, the time when many people go home or go out for fun. MTA, on the other hand, shows a different behaviour during the day, with different busy moments.

  
\begin{figure}%
\centering
\subfigure[Uber trips.]{%
\label{fig:a}%
\includegraphics[height=2in, width=3.5in]{UBER_trips_only_weekdays_rush_and_non_rush.png}}%
\qquad
\subfigure[Yellow cab trips]{%
\label{fig:b}%
\includegraphics[height=2in, width=3.5in]{yellowTripsRushHours.png}}%
\qquad
\subfigure[Green cab trips]{%
\label{fig:c}%
\includegraphics[height=2.4in, width=3.5in]{green_trips_month_rush.png}}%
\qquad
\subfigure[MTA trips]{%
\label{fig:d}%
\includegraphics[height=2in, width=3.5in]{mta_new_entries_orange_rush_hour.png}}%
\caption{Has the increase of Uber trips affected the number of trips of Yellow cab, Green cab, and MTA? Orange boxes represent the number of trips in rush hours and blue ones correspond to non-rush hours. }
\label{fig:boxTrips}%
\end{figure}





%Figure \ref{fig:boxDistancesAndPrice} compares the number of trips made by Uber with the monthly average travel distance covered by Yellow cabs. The aim of this comparison was to analyze if the increase of Uber trips affected the average travel distance of   plots, it is possible to see that from 2015, Uber is widely used for long distance trips, in contrast to Yellow cabs. On the other hand, the average travel distance of Yellow cabs experienced an increase in April 2015. 


\begin{figure}%
\centering
\subfigure[Uber trips.]{%
\label{fig:a}%
\includegraphics[height=2in, width=3.5in]{UBER_trips_only_weekdays_rush_and_non_rush.png}}%
\qquad
\subfigure[Yellow cab trips]{%
\label{fig:b}%
\includegraphics[height=2in, width=3.5in]{yellow_trips_month_weekend.png}}%
\qquad
\subfigure[Green cab trips]{%
\label{fig:c}%
\includegraphics[height=2.4in, width=3.5in]{green_trips_month_weekend.png}}%
\qquad
\subfigure[MTA trips]{%
\label{fig:d}%
\includegraphics[height=2in, width=3.5in]{mta_entries_month_weekend.png}}%
\caption{Has the increase of Uber trips affected the number of trips of Yellow cab, Green cab, and MTA? Orange boxes represent the number of trips in weekends and blue ones correspond to weekdays. }
\label{fig:boxTripsWeekends}%
\end{figure}






\begin{figure}%
\centering
\subfigure[Uber trips.]{%
\label{fig:a}%
\includegraphics[height=2in, width=3.5in]{uber_trips_hour_week.png}}%
\qquad
\subfigure[Yellow cab trips]{%
\label{fig:b}%
\includegraphics[height=2in, width=3.5in]{yellow_trips_hour_week.png}}%
\qquad
\subfigure[Green cab trips]{%
\label{fig:c}%
\includegraphics[height=2.4in, width=3.5in]{green_trips_hour_week.png}}%
\qquad
\subfigure[MTA trips]{%
\label{fig:d}%
\includegraphics[height=2in, width=3.5in]{mta_entries_hour_week.png}}%
\caption{Behaviour of the number of trips per hour. }
\label{fig:boxTripsWeekends }
\label{fig:boxTripsHour}%
\end{figure}



On the other hand to analyze Hypothesis \textbf{H4}, the plots of Figure \ref{fig:boxDistancesPrice} were created. Figure \ref{fig:boxDistancesPrice_a} and \ref{fig:boxDistancesPrice_c} proof what was mentioned in in Section \ref{intro_background}, the price of a trip by Green cab is cheaper than by Yellow cabs. However, the plots show that the price of Green and Yellow cab trips have not reduced as we thought.


Figures \ref{fig:boxDistancesPrice_b} and \ref{fig:boxDistancesPrice_d} analyze the traveled distances of yellow and green cabs. The plots show that they have not vary significatively, as we stated in Hypothesis \textbf{H1}.


\begin{figure}%
\centering
\subfigure[Yellow cabs. Average price per distance (USD/mile)]{%
\label{fig:boxDistancesPrice_a}%
\includegraphics[height=2in, width=3.5in]{yellow_avgamountperdistance_rush_month_boxplot.png}}
\qquad
\subfigure[Yellow cabs. Average distance per month]{%
\label{fig:boxDistancesPrice_b}%
\includegraphics[height=2in, width=3.5in] {yellow_avgdistance_month_boxplot}}
\subfigure[Green cabs. Average price per distance (USD/mile)]{%
\label{fig:boxDistancesPrice_c}%
\includegraphics[height=2in, width=3.5in]{green_avgamountperdistance_rush_month_boxplot.png}}
\qquad
\subfigure[Green cabs. Average distance per month]{%
\label{fig:boxDistancesPrice_d}%
\includegraphics[height=2in, width=3.5in]{green_avgdistance_month.png}}%
\caption{Distance analysis for Yellow and Green cabs. Images a and c shows the monthly average price per distance. Images b and d show the average distance per month.}
\label{fig:boxDistancesPrice}%
\end{figure}






\subsubsection{Heat Maps}
\label{secc:heatMaps}
In this section different maps were created to analyze which areas of the city, in terms of NTAs, are the ones with less coverage; or what are the most common transportation choices of people, per NTA. Two different maps were created. The first one, Figure \ref{fig:heatMapTripsPick}, shows the number of pick-ups and and the pickup zone, of the different transportation options. On the other hand, Figure \ref{fig:heatMapTripsDrop}, shows the number of drop-offs and the drop-off zones.

Table \ref{tab:dataset} shows the Links to access the interactive Heat Maps. The maps show the number of trips per NTA, in the different transportation options.

\begin{table}[h]
\begin{center}
\begin{tabular}{|c|c|}
\hline
\textbf{Figure Name}     & \textbf{Link to Map}  \\ 
\hline
Trips NTA Green Dropoff Map     & \href{https://marioceron-case-51.s3.amazonaws.com/datathon_html/trips_nta_green_dropoff_map.html}{Link}   \\
Trips NTA MTA Dropoff Map       & \href{https://marioceron-case-51.s3.amazonaws.com/datathon_html/trips_nta_mta_dropoff_map.html}{Link}   \\
Trips NTA Yellow Dropoff Map    & \href{https://marioceron-case-51.s3.amazonaws.com/datathon_html/trips_nta_yellow_dropoff_map.html}{Link}  \\
Trips Population NTA Green Map  & \href{https://marioceron-case-51.s3.amazonaws.com/datathon_html/trips_pop_nta_green_map.html}{Link}      \\
Trips Population NTA MTA Map    & \href{https://marioceron-case-51.s3.amazonaws.com/datathon_html/trips_pop_nta_mta_map.html}{Link}       \\
Trips Population NTA Uber Map  & \href{https://marioceron-case-51.s3.amazonaws.com/datathon_html/trips_pop_nta_uber_map.html}{Link}      \\
Trips Population NTA Yellow Map & \href{https://marioceron-case-51.s3.amazonaws.com/datathon_html/trips_pop_nta_yellow_map.html}{Link}\\

Total Pick Ups per NTA & \href{https://marioceron-case-51.s3.amazonaws.com/datathon_html/trips_nta_total_pickup.html}{Link}\\
Total Drop Offs & \href{https://marioceron-case-51.s3.amazonaws.com/datathon_html/trips_nta_max_coverage_dropoff.html}{Link}\\
Pick Ups & \href{https://marioceron-case-51.s3.amazonaws.com/datathon_html/trips_nta_max_coverage.html}{Link}\\
Drop Offs & \href{https://marioceron-case-51.s3.amazonaws.com/datathon_html/trips_nta_max_coverage_dropoff.html}{Link}\\
 \hline
\end{tabular}
 \caption{Links to access interactive Heat Maps. The maps show the number of trips per NTA, in the different transportation options.}
 \label{tab:dataset}
\end{center}
\end{table}



\begin{figure}%
\centering
\subfigure[Uber.]{%
\label{fig:a}%
\includegraphics[height=2.5in, width=3in]{uberNormPopulation_pickup.png}}%
\subfigure[Yellow cab]{%
\label{fig:b}%
\includegraphics[height=2.5in, width=3in]{yellowNormPopulation_pickup.png}}
\qquad
\subfigure[Green cab ]{%
\label{fig:c}%
\includegraphics[height=2.5in, width=3in] {greenNormPopulation_pickup.png}}
\subfigure[MTA ]{%
\label{fig:d}%
\includegraphics[height=2.5in, width=3in] {mtaNormPopulation_pickup.png}
}%
\caption{Heat maps of Pick Ups. The maps show the number of trips per NTA, in the different transportation options. The data has been normalized by the NTA population.}
\label{fig:heatMapTripsPick}%
\end{figure}


\begin{figure}%
\centering
\subfigure[Yellow cab trips]{%
\label{fig:b}%
\includegraphics[height=2.5in, width=3in]{yellowNormPopulation_dropoff.png}}
\qquad
\subfigure[Green cab trips]{%
\label{fig:c}%
\includegraphics[height=2.5in, width=3in] {greenNormPopulation_dropoff.png}}
\subfigure[MTA trips]{%
\label{fig:d}%
\includegraphics[height=2.5in, width=3in] {mtaNormPopulation_dropoff.png}
}%
\caption{Heat maps of Drop offs. The maps show the number of trips per NTA, in the different transportation options. }
\label{fig:heatMapTripsDrop}%
\end{figure}





Figure \ref{fig:heatMapTotalPickDrop} shows the location and the total number of pick ups and drop offs, of Green, Yellow, UBER (only pick up information was available) and MTA transportation means. The data is normalized by the population of each MTA. These maps start showing areas where public transportation is not widely used.


\begin{figure}%
\centering
\subfigure[Total Pick Ups per NTA]{%
\label{fig:heatMapTotalPickDrop_a}%
\includegraphics[height=2.5in, width=3in]{totalNormPickUps.png}}
\qquad
\subfigure[Total Drop Offs]{%
\label{fig:heatMapTotalPickDrop_b}%
\includegraphics[height=2.5in, width=3in] {totalNormDropOffs.png}}
\caption{Analyzing the total number of Pick Ups and Drop Off of all the transportation options. }
\label{fig:heatMapTotalPickDrop}%
\end{figure}



On the other hand, Figure \ref{fig:maxMapTotalPickDrop} shows per NTA, which is the transportation option more used in the area. Analyzing the map that appears on the left (Pick ups), it can be seen that in the outer areas of the city, UBER is widely used. However, MTA is the prefer transportation option in most of the city. The latter coincides with what was found in Section \ref{intro_background}: NYC is famous because people does not like to own a car, they prefer to use public transportation.

The right figure shows the drop off map. It is important to remember that there is no drop off information of UBER, and this is why in the outer areas, it is possible see that yellow and green cabs options are widely used. This map is different from what we expected. The reason is that according to the literature, Yellow cars do not like to move far from Manhattan. However, the map shows that they are moving far from it.


\begin{figure}%
\centering
\includegraphics[height=3.5in, width=7in]{maxOptionMap.png}
\caption{Most frequently transportation mean per NTA. The maps show the number of trips per NTA, in the different transportation options.}
\label{fig:maxMapTotalPickDrop}%
\end{figure}


%\section{Statistical Analysis and Modeling}
\label{sec:StatAnalisys}

The Exploratory Data Analysis conducted in previous section, hypothesis \textbf{H2} was confirmed (Due to UBER incursion in the city, the \textbf{number of trips} of yellow cabs was reduced) , and hypothesis \textbf{H4} was rejected (Due to UBER incursion in the city, the price of yellow cabs trips have decreased).

In this section we will conduct a statistical analysis to hypothesis 1 \textbf{H1} (due to UBER incursion in the city, the \textbf{travel distances} of yellow cabs trips increased), and we will provide more details to answer hypothesis 3 \textbf{H3} (The areas of the city where there is \textbf{no public transportation} coverage, correspond to areas with \textbf{low income families}). 

The notebooks used to conduct the analysis can be found in the following link: 
\href{https://marioceron-case-51.s3.amazonaws.com/datathon_html/Models_Demographics.html}{Link}


\subsection{Analysis of Hypothesis 1 \textbf{H1}. Due to UBER incursion in the city, the \textbf{travel distances} of yellow cabs trips increased}
We conducted a ttest between the distance of the trips that took place between 2014-04 and 2014-06.  The null hypothesis that is tested is that there is no change in the mean distance in both periods.

\begin{table}[h]
\begin{center}
\begin{tabular}{lclcl}
\hline
\textbf{Hypothesis Testing}  & \textbf{P-Value}   \\
\hline
Green Cabs rush & 0.0 \\
 Green Cabs weekends  &  0.0  \\
Green Cabs  & 0.0078 \\
\hline
\end{tabular}
\caption{ttest Green cabs.}
\label{tab:ttestGreen}
\end{center}
\end{table}


From Table \ref{tab:ttestGreen}, we can conclude that the null hypothesis is rejected. 

\begin{table}[h]
\begin{center}
\begin{tabular}{lclcl}
\hline
\textbf{Hypothesis Testing}  & \textbf{P-Value}   \\
\hline
Yellow Cabs weekends/weekday & 0.181 \\
Yellow Cabs rush-non rush-non	  &  0.905  \\
\hline
\end{tabular}
\caption{ttest Yellow cabs.}
\label{tab:ttestYellow}
\end{center}
\end{table}

From Table \ref{tab:ttestYellow}, we can conclude that we fail to reject the null hypothesis. There is no enough statistical evidence to assert that the distance of the trips that took
place between 2014-04 and 2014-06 is DIFFERENT from the ones between 2015-04 and 2015-06.



\subsection{Analysis of Hypothesis 3 \textbf{H3} The areas of the city where there is \textbf{no public transportation} coverage, correspond to areas with \textbf{low income families} 
}

The maps created in Section \ref{secc:heatMaps} provided a clue of the behaviour in each NTA of the different analyzed trasnportation options. The variable ``yellow\_indicator" was created. It allows us to know in which NTAs, public transportation has not an adequate coverage, or in which ones public transportation is not commonly used. Both analyzed with the total number of trips per NTA. Details of how this variable was created are provided in Section \ref{subsubsec:featureEng}.

With the variable ``yellow\_indicator" defined per NTA (1 if not many trips started in that NTA, otherwise 0), demographic information was analyzed to know if there is a pattern in the income of the people that live in those areas. Figure \ref{fig:incomesNTA} shows the violin and box plots of the mean income vs the indicator of used of public transportation. From the plots we can see that most of the people that do not frequently use public transportation, do correspond to people with high incomes. Both plots show that we have some outliers in the data, that could lead to miss-interpretation of the information. In the areas where people use public transportation more frequently, live people with very high incomes.  

Excluding people with very high incomes from the analysis, we can reject the stated hypothesis. Digging into more details, we found that the people with the highest rent live in Manhattan, which correspond to one of the areas where public transportation is commonly used.

\begin{figure}%
\centering
\subfigure[Violin plot ]{%
\label{fig:heatMapTotalPickDrop_a}%
\includegraphics[height=3in, width=3in]{violinMeanIncome.png}}
\qquad
\subfigure[Box plot ]{%
\label{fig:heatMapTotalPickDrop_b}%
\includegraphics[height=3in, width=3in] {boxplotMeanIncome.png}}
\caption{Analyzing Mean income between NTAs that frequently use public transportation (yellow indicator = 0) vs. the ones that do not frequently use it  (yellow indicator = 1)}
\label{fig:incomesNTA}%
\end{figure}



Finally, Figure \ref{fig:age_incomesNTA} shows the comparison between the different age ranges and income ranges of the NTAs that have a yellow\_indicator = 0 and yellow\_indicator = 1. From the figure we can see the following pattern of people that live in the NTAs with yellow indicator = 1 :

\begin{itemize}
\item Most of the people that live in those areas are adults over 40 years old, some families with children.
\item Most of them with higher incomes ($50000<$ incomes $< 200.000$ USD) than the ones that live in the zones with yellow\_indicator = $0$. Excluding the ones with incomes $> 200.000$

\end{itemize}

 
\begin{figure}%
\centering
\subfigure[Distribution of age between the NTAs with yellow\_indicator = 0 and yellow\_indicator = 1]{%
\label{fig:heatMapTotalPickDrop_a}%
\includegraphics[height=4in, width=7in]{age_norm}}
\qquad
\subfigure[Distribution of income between the NTAs with yellow\_indicator = 0 and yellow\_indicator = 1]{%
\label{fig:heatMapTotalPickDrop_b}%
\includegraphics[height=4in, width=7in] {income_norm}}
\caption{Analyzing the behaviour of age and income between the the NTAs with yellow\_indicator = 0 and yellow\_indicator = 1}
\label{fig:age_incomesNTA}%
\end{figure}


\subsection{Model fitting}

We fit a model a random forest to predict the Yellow Indicator, based on some demographic information. The idea was to be able to create a model trained with data from 2014 and 2015 that could be used to create a new map projected in 2019. The latter with the aim of analyzing if \textbf{no changes} are done in terms of mobility, and population grows and demographic information changes, how do those changes affect the coverage. 

The model was created with the following variables. The dependent variable is the yellow\_indicator. The independent variables the different age ranges (normalized by population), and the different income ranges (normalized by population). The data was split 70\% for training and 30\% for testing. We obtain an 87.72\% of accuracy and 82.35\% of precision. Figure \ref{fig:randomForest} shows a histogram of the results with the test set. The histogram was created plotting the differences between the real values and the predicted  ones. 

\begin{figure}%
\centering
\includegraphics[height=2.5in, width=4in]{random_forest_yellow-indicator.jpg}
\caption{Random forest predictions for the yellow\_indicator. The plot shows the differences between the real values and the predicted ones.}
\label{fig:randomForest}%
\end{figure}


%\input{sections/statisticsConclusions}


%\begin{table}[h]
%\begin{center}
%\label{tabDataset}
%\begin{tabular}{|c|c|c|c|c|}
%\hline
%\textbf{Data Name} & \textbf{Description} & \textbf{Type} & \textbf{Number} & \textbf{Database}   \\
%\hline
% Event name: & type of disaster (flooding,XX)& categorical &S &  UNGRD\\
% Date: & incident date & numerical &XXX &  UNGRD/DANE  \\
% Code: &disaster ID & numerical &XXX &  UNGRD/DANE  \\
% Municipality Code: & Divipol code & numerical &XXX &  UNGRD/DANE  \\
% Dead: & Deads per incident & numerical &XXX &  UNGRD  \\
% Wounded: & Wound per incident & numerical &XXX &  UNGRD  \\
% Disappeared: & Disappeared per incident & numerical &XXX &  UNGRD  \\
%  Affected: & Affected people & numerical &XXX &  UNGRD  \\
%   Affected families: & Affected people & numerical &XXX &  UNGRD  \\
%\hline
%\end{tabular}
% \caption{Summary of the main information available to develop the project.}
%\end{center}
%\end{table}


\bibliographystyle{abbrv}
\bibliography{simple}

\end{document}
This is never printed
