
\documentclass[11pt]{article}

\usepackage[margin = 0.7 in]{geometry}
\usepackage[pdftex]{graphicx}
\graphicspath{{images/}}
\usepackage{subfigure}
\usepackage{booktabs}
\usepackage{multirow}
\usepackage[table]{xcolor}
\usepackage{array}
%\usepackage{amsmath}
\usepackage{booktabs}
\usepackage{multirow}
\usepackage{amsmath}
\usepackage{url}

\hyphenation{op-tical net-works semi-conduc-tor re-co-mmen-ded}


\title{Project Group 3}
\author{ Johnathan Salamanca, Mario Cer\'{o}n, \\
Carol Martinez, Javier Cocunubo, Jairo Nino, Alvaro Munoz
 }
\date{\today}

\begin{document}
\maketitle

%\begin{abstract}

%In this document \ldots
%\end{abstract}


\section{Introduction}
\label{intro}
This document describes the process followed to define, analyze and answer  the questions of the Datathon project. The data provided correspond to datasets from 2014 and 2015 of four different transportation means: Yellow cabs, Green cabs, UBER, and MTA. Additionally, one dataset provides weather information, and another one provides demographic information of the boroughs. 


\subsection{Background Information}
\label{intro_background}

New York city is the most populous city in the United States \cite{wikiNYC} with around 8.6 million inhabitants. It is expected to reach 9 million by 2040 \cite{growthNYC}, with Bronx being the borough with the highest increase in population with 14\% (between 2010 and 2040). Average Travel speeds in New York City is 10 mph, 4 mph or less in Brooklyn and Queens, and over 3 mph NYC Midtown. 


In terms of transportation, according to \cite{trafficNYC2}, New York City is the second most congested city in the USA and number 42 in the world. It has one of the largest and oldest (1904) subway systems in the world. Due to its congestion, New York City inhabitants prefer to use public transportation ($67.2\%$ of workers commuting to work by this means in 2006 \cite{wikiNYC})

However, MTA the worst commutes in the world \cite{commuteWorld} conmute time of 35.6 minutes on average commuting has been linked to obesity, stress, anxiety, depression, higher blood pressure, higher rates of divorce, neck and back pain and shorter lifespans.





%In Midtown taxi trips are short distances... comparable ttrips by citi bikes are generally faster and less expensive...  between 1-1.5 miles are more than 5 minutes faster and 10 USD cheaper than taxi.. 
%
%Recent Travel trends:
%- Number of trips of yellow cabs drop by XXX
%- number o for hire vehicle registrations has grown (yellow cabs green cabs, black cabs, private cabs companies)..  72000 in 2015
% - MTA 
% -Higher speeds LESS dense residential areas of eastern queens and the northern sections of the Bronx Staten Island
% - Average taxi speed.
% 
% 
% between the 5 most congested cities in the USA one of the 
%It has one of the largest subway systems in the world. It is also famous because people does not like to own a car they prefer to use public transportation.. Highest rate of public transportation use of any american city $67.2\%$ of workers commuting to work by this means in 2006.. Longest mean travel time for commuters. Half of all households do not own a car. In manhattan around 75\% 

%2014 8.521 mllion
%2015  8.522 million
%2019  8.398.748 million

%New York City's population is expected to reach 9 million by 2040 \cite{growthNYC}, based on recent projections created by the city. Among the five boroughs, the Bronx's growth is projected to be the highest at 14\% between 2010 and 2040. On the flip side, Manhattan is expected to grow by 6.7\% by 2040. Median age are 36.2 years between male and female.. rate of home ownership 32.6\% Percentage in POVERTY POWERTY RATE.. EMPLOYMENT STATUS

%If these projections are accurate, Brooklyn will extend its lead over Queens as the largest borough in New York City, growing to nearly 3 million by 2040.
%
%
%NYC mass transit..  
%
%NYC This city houses the highest number of millionaires and billionaires in the world.
%
%High rate of publich transit use
%a city is the most mwith XXX inhabitants [ref]. It is commontly known for its transportation innovations 
%NYC is the third most traffic congested city in the world \cite{trafficNYC}
%
%
%3.7 million people were employed in New York City
%Manhattan 56\% of all jobs
%Of those working in Manhattan, 30% commute from within Manhattan, while 17% come from Queens, 16% from Brooklyn, 8% from the Bronx, and 2.5% from Staten Island. Another 4.5% commute to Manhattan from Nassau County and 2% from Suffolk County on Long Island, while 4% commute from Westchester County. 5% commute from Bergen and Hudson counties in New Jersey.
%NYC 8.5 million inhabitants.

%\textbf{future of NYC}
%Traffic has been always a problem in NYC 

%In terms of Yellow cabs, the following list summarizes key points of this transportation mode. 
The following lists summarize key points of the different transportation means used in the Datathon. \\

\textbf{Yellow Cabs}

\begin{itemize}
\item Mostly located in Manhattan
\item It is very difficult to get a taxi out of Manhattan, especially in rush hours.
\item There are not many yellow cabs 130000, which are not enough.
\item They can operate midtown and lower Manhattan and airports
\item Rarely pick up outside manhattan
\end{itemize}



\textbf{Green Cabs}
\begin{itemize}
\item Where created to standardize street hails outside of NYC. They operate in Brooklyn, Queens, The Bronx, and upper Manhattan. 
%\item Created to provide more access to metered taxis. Less expensive than livery cars.
%\item They operate Manhattan Bello 110th St on the west side and below 96th street on the east side or at either la guardia or JFK airports.

\item Airports: they can drop-off but not pick-up unless sent by a dispatcher
%\item They can \textbf{be on call by dispatcher}
\item They are not allowed to stop in the South of the upper west and upper east sides.
%\item The permits of Green cabs are cheaper than yellow cabs and easier to acquire. Cab license affordable for drivers.
\item Rides cheaper than Yellow cab rides
\item For drivers green cabs was a way to get money without the pressure that yellow drivers have.
\item Green drivers  \textbf{are also drivers or UBER}
\item 1/3 pick ups from Brooklyn,  1/3 Northern Manhattan, 1/3 Queens, a few in Bronx and Staten Island \\
\end{itemize}

%\subsubsection{Uber}


\textbf{UBER}

\begin{itemize}
\item Started in 2011 in Manhattan but expanded at the same time green.
\item Uber has made yellow cabs steady but has impacted green that were just started 
\item Uber has made yellow cabs steady but has impacted green that were just started 
\item Connects drivers with more rides
\item May 2015 busiest month on record.
\end{itemize}



According to the mobility report of 2016 \cite{mobilityReportNYC2}, New York City is growing in jobs, residents and visitors. Its transportations modes have extend to mass transit, walking, and cycling. New York advances focus on applying technology to real-time traffic management, Select Bus Service routes, reducing travel times, Expanding the city's bicycle lane network, improving pedestrian access to transit. It is still required to invest in ways to keep the city moving. 


Despite the incursion of new transportation means e.g. Green Cabs, UBER, among others, it is still difficult to catch a taxi from outer Boroughs to Manhattan. Lower access to legal taxi rides for people in outer Boroughs.
%\item Green cabs try but never fulfilled their promise


%\end{itemize}


\subsection{Topic Question}

\textbf{General question}: \\

%\begin{center}
%\tcbox[top=15pt,left=15pt,right=15pt,bottom=15pt, on line, fontupper=\Large]{ \textit{Is public transportation coverage in New York City \newline 
%\textbf{equally attending} all areas?}}
%\end{center}

\begin{tcolorbox}[colback=white,top=6pt,bottom=6pt, breakable,arc=3pt,
  outer arc=3pt]
\begin{center}
 \textit{Is public transportation coverage in New York City \\
\textbf{equally attending} all areas?\\}
\end{center}
\end{tcolorbox}

\vspace{0.5cm}

\textbf{Exploratory questions}: 
\begin{itemize}
\item What are the patterns related to unattended areas of public service?
\item Is there are relationship between demographic information and peoples' choices of public transportation?
\end{itemize}




%\begin{center}
%\begin{myboxi}[General question]
%\center
%\textit{Is public transportation coverage in New York City equally attending all areas?} \\ 
%\end{myboxi}
%\end{center}



\section{Multiple Problem Versions (week 3)}
\label{sec:app}
%All versions should naturally build on top of each other
%? V1 should be pretty easy and reasonable
%? The last version can be moonshot - the idea is that you must implement
%for V1 first, then V2, etc. so as to guarantee some sort of finished build by
%the end of Week 10. If you can get to V2, V3, etc. by the end, great, if not,
%at the very least you have something done that you can present)
%Business Problem. Your main task is to explore the data and identify patterns of crime in Chicago, and come up with strategies to efficiently deploy your workforce to fight crime.

Based on the previously presented information different questions have arisen. 
\begin{enumerate}

%\item{In 2012 was created in Colombia the National System of Risk Management. Based on the available datasets is it possible to analyze and find patterns that show (\textbf{V3}), 
%\begin{itemize}
%\item how does the risk map of Colombia changed after the creation of this system? 
%\item How does the specific infrastructure affects the impact of the specific event? 
%\end{itemize}
%}

\item{There is a disproportionate impact of similar events among Colombia's municipalities, given by disparities in available infrastructure and first response resources  (\textbf{V2}).}


\item{Is it possible to analyze a specific event (disaster) and show how does the same event  affects different zones of the country? Based on that, we can analyze (\textbf{V1}):
\begin{itemize}
\item Are there factors that make some zones more vulnerable than others?
\item How does the specific infrastructure affects the impact of the specific event? 
\end{itemize}
}
\end{enumerate}

%\subsection{Question}


\section{Datasets sourced (week 3)}
\label{sec:app}

The main dataset used in the project is from the Colombia Risk of Disaster Management Unit (Unidad de Gesti\'{o}n de Riesgos y Desastres) UNGRD \cite{datasetUNGRD}. The dataset contains information about  the risk management associated with natural phenomena, socio-natural, technologic, and human-based non-intentional incidents reported in Colombia in the last 10 years (38626 records). Some of the fields found in the dataset are: Date, Department, Municipality, Event Name, Code, Dead, Wounded, Disappeared, Affected People, Affected Families, Affected Houses, among others. 

The team will also use a dataset from the National Administrative Department for Statistics DANE. It is a time series between 1985 to 2020 and contains information, per department code about \cite{DANE} .

Both datasets contain ``DIVIPOL'' codes, which is the codification of the Politica-Administrative Division of Colombia (Codification of the departments, ). Figure \ref{fig:divipol} describes the meaning of the code. The first two numbers correspond to the department, followed by the Municipality Code and the Populated Center [\cite{divipola}].
 

\begin{figure}[!ht]
        \center{\includegraphics[width=0.6\textwidth]
        {divipolaExplanation}}
        \caption{Explanation of ``DIVIPOL'' code. The codes provide information of the Politica-Administrative Division of Colombia. Image taken from \cite{divipol}}
        \label{fig:divipol}
      \end{figure}
      

Table \ref{tabDataset} summarizes the information of the datasets. 

\begin{table}[h]
\begin{center}
\label{tabDataset}
\begin{tabular}{|c|c|c|c|c|}
\hline
\textbf{Data Name} & \textbf{Description} & \textbf{Type} & \textbf{Number} & \textbf{Database}   \\
\hline
 Event name: & type of disaster (flooding,XX)& categorical &S &  UNGRD\\
 Date: & incident date & numerical &XXX &  UNGRD/DANE  \\
 Code: &disaster ID & numerical &XXX &  UNGRD/DANE  \\
 Municipality Code: & Divipol code & numerical &XXX &  UNGRD/DANE  \\
 Dead: & Deads per incident & numerical &XXX &  UNGRD  \\
 Wounded: & Wound per incident & numerical &XXX &  UNGRD  \\
 Disappeared: & Disappeared per incident & numerical &XXX &  UNGRD  \\
  Affected: & Affected people & numerical &XXX &  UNGRD  \\
   Affected families: & Affected people & numerical &XXX &  UNGRD  \\
\hline
\end{tabular}
 \caption{Summary of the main information available to develop the project.}
\end{center}
\end{table}
      



%  (data of ):
%http://portal.gestiondelriesgo.gov.co/Paginas/Consolidado-Atencion-de-Emergencias.aspx
%We plan to cross data from the previous dataset with data from the DANE, IDEAM
%We may require datasets to extract data of:
%Hospitals available in the area
%Schools
%Emergency services available.
%Demographic data



%Santos talks about Colombia Progress on Risk Management https://www.unisdr.org/archive/58870

%? ?El Ni�o? and particularly the La Ni�a 2010-2011, which generated a national emergency situation never before seen in the country, affecting nearly 765 municipalities in Colombia, ?. After these phenomena, the government decided to improve the gathering information systems (info taken from
%https://library.wmo.int/doc_num.php?explnum_id=4759 )


\section{Project scoping plan/proposal written (week 3)}
\label{sec:app}

\subsection{Project scopes}

\begin{itemize}
\item The Government Officials (at all levels) are our main stakeholders
\item Boundaries of the project:
\begin{itemize}
\item We do not offer forecasts or modeling/ infer data.
\item We only show metrics of impact of disasters at municipal level (not pin point to specific disaster event)
\item We do not offer recommendations, only do support to decision making process for the stakeholders.
\end{itemize}
\item Note the biggest risks to the successful completion of the project.
\begin{itemize}
\item Data quality issues in the datasets (non fixed easily)
\item The final comparison product does not have sufficient enough information consolidated to explain the main problem to be resolved.
\end{itemize}
\end{itemize}


\subsection{Project plan}

\textbf{Problem}: there is a disproportionate impact of similar events among Colombia?s municipalities,
given by disparities in available infrastructure and first response resources. Impact is defined in terms of: Inhabitants affected/Thousand Inhabitants, percentage of households affected, Deceased/ Thousand Inhabitants.\\

\textbf{Expected Deliverable}:\\
A dynamic map of Colombia, visually delivering the impact metric (or metrics) at the municipal level for a given category of events. Ability to display complementary metrics of interest for specific locations (utilities, healthcare facilities, first-responders facilities, etc).\\

\textbf{How do we get there?}:

\begin{itemize}
\item Get datasets, cleaned, wrangled and analyzed. 
\item Make the relevant joins on jupyter notebooks (localy), merge and generate bases to have the data (modeling, generate parameters and boundaries)
\item Load the data at the database in the cloud (RDS on AWS) and the instance (EC2) for host the back and front end. Install and review the environment.
\item Make the back end: review databases, environment and set jupyter / notebooks for run the cloud
\item Make the front end : Interactive Colombian Map in Dash and tested on AWS.
\item Make the final presentation
\item Make the final report
\end{itemize}

%\section{Application Overview}
\label{sec:appOver}


\subsection{Users}
Any person, Government officials, or someone in the private sector who wants to read and understand the basic risk profile of their region and make decisions with that information.

\subsection{Architecture}

Figure \ref{fig:archi} shows the architecture of the proposed solution including the elements of the application at component level and its connections at high level (see deployment diagram). Additionally, it shows the application elements used for the Front and Back End. The figure also shows the names of the technologies used hosted on AWS cloud, i.e.: (Python, dash and libraries).   

The following is the list AWS components used in the project: 
\begin{enumerate}
\item The machine who host the solution (Elastic Compute Cloud - EC2). 
\item The Database  (Relational Database Service -RDS).
\item The storage for the datasets and GeoJson files for Colombia on the service  (Simple Storage Service - S3) to save these files.
\item The Security group for these services talks with each other and have access from the internet as well.
\item The remote DNS (Domain Name Service) to have a friendly URL for the application on the Apache Web server.
\item A remote code repository (hosted by github). It is used for hosting the source code and documentation. 

\end{enumerate}


\begin{figure}%
\centering
\subfigure[Deployment diagram]{%
\label{fig:first}%
\includegraphics[height=3in, width=4.5in]{../../../aws_diagrams/AWS_Deployment_Diagram_project_group_03}}%
\qquad
\subfigure[Component diagram]{%
\label{fig:second}%
\includegraphics[height=4in, width=6in]{../../../aws_diagrams/AWS_Component_Diagram_project_group_03}}%
\caption{Project architecture.}
\label{fig:archi}%
\end{figure}





\subsection{Front End Design}

The following is the link of the web-page created for the project. \\
 \url{http://ds4a-colombia-group03.tk/}
%\item \href{https://marioceron-case-51.s3.amazonaws.com/final_project/front_end/index.html}{http://ds4a-colombia-group03.tk/} 

In general terms, the proposed information system will provide a dynamic map of Colombia with:

\begin{enumerate}
\item Impact variables such as Deaths, People and Houses affected.
\item A map (Main View) that will show the index adjusted by capacity ( \'Indice Ajustado por Capacidades). By default, the map starts with the country view with political division (departamento).
\item A timeline slider will facilitate the visualization of the index using time intervals. 
\item Considering the established association between extreme temperatures and the frequency of hydro-meteorological events, a projected extreme-temperature indicator for the 100 most vulnerable municipalities with 3 data points: Indicator value at Time 0 (1998), Time 1 (2018) and Time 3 (projected 2040). The indicator corresponds to the extreme temperature projection made by Climate Impact Lab for the number of days a year that register temperatures above 32 degrees Celsius.

\item The option of showing political divisions views (vista de departamento). The user can select the small political subdivision (municipio) and the graphics on the right side of the dashboard screen are updated.
\end{enumerate}




\begin{figure}%
\centering
\subfigure[Presentation page]{%
\label{fig:first}%
\includegraphics[height=4in, width=6in]{01_Home_ds4a-colombia-group03-tk.png}}%
\qquad
\subfigure[Main page]{%
\label{fig:second}%
\includegraphics[height=4in, width=6in]{02_Home_mapa.png}}%
%\subfigure[]{%
%\label{fig:second}%
%\includegraphics[height=2in, width=3in]{riesgo6}}%
\caption{Front End: Integrated Map of Risk Information in Colombia. The main page (Figure \ref{fig:first}) will show the map of Colombia with the Risk Index, and will offer interactive options to get detail information of selected Municipalities, risk index and a projected temperature indicator.}
\label{fig:muckup}%
\end{figure}


%\section{Technical Information}
\label{sec:Tech}

\subsection{AWS-hosted database}

Figure \ref{fig:er_model} shows the entity relationship model of the project. The information contained in the model is:

\begin{enumerate}

\item Disasters : information of disasters of Colombia from 1998 to 2017, with the following fields: date,  Colombian political division, number of disasters, death, injuries, missing people, families, houses, public and education services infrastructures and some economical  information. (table disasters)
\item Events: name and category of the event (table eventos)
 
\item Political division of Colombia (divipola): ID of division, subdivision and name of both (divipola table)

\item Population estimates: relates population by period, age groups, political division, id, and gender

\item Weather estimates:  weather information from 1990 to 2018 (minimal temperatures, maximal temperatures, precipitations) (tables: historico\_cond\_metereologicas, load\_mintemp, load\_maxtemp, load\_precipitaciones)	


\item Load tables: temporal tables for loading disasters, political division, population (tables: load\_disasters, load\_divipola, load\_populations)

\item Views: wv\_disasters (summary table for disasters). 

\end{enumerate}

The SQL script for the creation of the database on AWS can be download from this link: \\ \url{https://marioceron-case-51.s3.amazonaws.com/final_project/Script_Desastres_DB.sql}{}

Full backup Database: \\ 
\url{https://marioceron-case-51.s3.amazonaws.com/final_project/Script_Desastres_DB_full.sql}{}


 
\begin{figure}[!htb]
\center{\includegraphics[width=0.95\textwidth]
{Project_Group03_ERModel}}
\caption{Entity relationship model.}
\label{fig:er_model}
\end{figure}

Figure \ref{fig:awsConnection} shows the database uploaded to AWS.

\begin{figure}[!htb]
\center{\includegraphics[width=0.95\textwidth]
{Desastres_AWS_Connection}}
\caption{AWS connection.}
\label{fig:awsConnection}
\end{figure}


Figure \ref{fig:databaseLoaded}  shows the database loaded to the AWS hosted database.

\begin{figure}[!htb]
\center{\includegraphics[width=0.95\textwidth]
{Desastres_data}}
\caption{Database loaded to AWS.}
\label{fig:databaseLoaded}
\end{figure}


The link between the front-end and the AWS-hosted databased was also stablished (see Figure \ref{fig:linkDASH_AWS}). 

\begin{figure}[!htb]
\center{\includegraphics[width=0.95\textwidth]
{link_dash_aws}}
\caption{Link between DASH and AWS stablished}
\label{fig:linkDASH_AWS}
\end{figure}






\bibliographystyle{abbrv}
\bibliography{project}

\end{document}
This is never printed
